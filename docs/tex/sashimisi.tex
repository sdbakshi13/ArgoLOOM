\documentclass[11pt]{article}

\usepackage[a4paper,margin=1in]{geometry}
\usepackage{amsmath,amssymb}
\usepackage{hyperref}
\usepackage{tcolorbox}
\usepackage{listings}
\usepackage{xcolor}

\hypersetup{
	colorlinks=true,
	linkcolor=blue,
	urlcolor=blue
}

\definecolor{codebg}{RGB}{245,245,245}

\lstset{
	backgroundcolor=\color{codebg},
	basicstyle=\ttfamily\small,
	breaklines=true,
	frame=single
}

\title{Documentation: Integrating \texttt{SASHIMI-SI} with \texttt{ArgoLOOM}}
\author{Internal Development Note}
\date{\today}

\begin{document}
	
	\maketitle
	
	\section*{Purpose of this Document}
	
	This note documents a \textbf{successful end-to-end integration} of the cosmological subhalo simulation code \texttt{SASHIMI-SI} with the orchestration framework \texttt{ArgoLOOM}.
	
	\begin{itemize}
		\item The goal is reproducibility for collaborators.
		\item This is a \textbf{success-case guide}: it shows what worked.
		\item It provides both minimal steps (to get running) and explanatory context.
	\end{itemize}
	
	\begin{tcolorbox}
		\textbf{Reading modes:}
		\begin{itemize}
			\item \textbf{Fast setup:} Follow code blocks only.
			\item \textbf{More details:} Read explanatory text.
		\end{itemize}
	\end{tcolorbox}
	
	---
	
	\section{Assumptions}
	
	We assume:
	
	\begin{itemize}
		\item \texttt{SASHIMI-SI} runs correctly standalone.
		\item \texttt{ArgoLOOM} runs correctly standalone.
		\item Python version tested: \textbf{3.11}. Other versions were not tested.
	\end{itemize}
	
	---
	
	\section{Environment Setup}
	
	\subsection{Conda Environment}
	
	\begin{lstlisting}
		conda create -n sashimi-si python=3.11 -y
		conda activate sashimi-si
	\end{lstlisting}
	
	---
	
	\subsection{Core Dependencies}
	
	\begin{lstlisting}
		pip install numpy==1.26.4
		pip install scipy==1.10.1
		pip install numexpr
		pip install tqdm
		pip install matplotlib
		pip install jupyter
		pip install openai
		pip install sentence-transformers
		
		# FAISS (may require conda-forge)
		conda install -c conda-forge faiss-cpu -y
	\end{lstlisting}
	
	\begin{tcolorbox}
		\textbf{Notes:}
		\begin{itemize}
			\item Mixed \texttt{pip} + \texttt{conda} install was used successfully.
			\item FAISS installation can vary by system.
			\item Python 3.11 compatibility was verified; other versions not tested.
		\end{itemize}
	\end{tcolorbox}
	
	---
	
	\section{Directory Structure and Code Edits}
	
	A representative working layout on the local machine is shown below. This is provided as a guideline for organizing the integration between \texttt{SASHIMI-SI} and \texttt{ArgoLOOM}.
	
	\begin{lstlisting}
		/Users/<username>/Documents/
		
			cosmowork/
				sashimi_si_project/
					argoloom_sashimi_adapter.py
					repo/sashimi-si/
					runs/
		
			GitHub/
				ArgoLOOM/
					argo-loom.py
					tools/cosmo/API_SASHIMI_subhalos.py
	\end{lstlisting}
	
	\begin{tcolorbox}
		Replace \texttt{<username>} with your machine username where necessary.  
		Absolute paths in scripts should be updated accordingly.
	\end{tcolorbox}
	
	\vspace{0.5cm}
	
	\subsection*{Repository Edits}
	
	All edits made to link \texttt{SASHIMI-SI} with \texttt{ArgoLOOM} are available in the public repository:
	
	\begin{center}
		\url{https://github.com/sdbakshi13/ArgoLOOM/tree/main}
	\end{center}
	
	Only the following components were modified or added:
	
	\begin{itemize}
		\item \texttt{argo-loom.py}  
		\begin{itemize}
			\item Addition of the \texttt{sashimi\_subhalos} tool schema.
			\item Tool execution block for dispatching SASHIMI runs.
		\end{itemize}
		
		\item \texttt{tools/cosmo/API\_SASHIMI\_subhalos.py}  
		\begin{itemize}
			\item New tool wrapper to interface ArgoLOOM with the SASHIMI adapter.
			\item Handles parameter passing and JSON output formatting.
		\end{itemize}
	\end{itemize}
	
	\begin{tcolorbox}
		No other core ArgoLOOM modules were modified.  
		This ensures minimal intrusion into the base orchestration framework.
	\end{tcolorbox}
	
	
	
	
	---
	
	\section{Conceptual Architecture (Skippable)}
	
	\begin{tcolorbox}
		\textbf{Skip if you only want setup instructions.}
	\end{tcolorbox}
	
	Pipeline flow:
	
	\[
	\text{ArgoLOOM} \rightarrow \text{Tool Call} \rightarrow \text{SASHIMI Adapter} \rightarrow \text{Catalog} \rightarrow \text{Summary JSON}
	\]
	
	Artifacts produced:
	
	\begin{itemize}
		\item \texttt{catalog\_raw.npz}
		\item \texttt{catalog\_SIDM\_survivors.npz}
		\item \texttt{summary.json}
	\end{itemize}
	
	---
	
	\section{SASHIMI Adapter}
	
	Location:
	
	\begin{lstlisting}
		cosmowork/sashimi_si_project/argoloom_sashimi_adapter.py
	\end{lstlisting}
	
	Purpose:
	
	\begin{itemize}
		\item Runs SASHIMI.
		\item Filters surviving subhalos.
		\item Writes catalogs + JSON summary.
		\item Prints JSON to stdout for ArgoLOOM ingestion.
	\end{itemize}
	
	(Use the working adapter script from development.)
	
	---
	
	\section{ArgoLOOM Tool Wrapper}
	
	Create:
	
	\begin{lstlisting}
		ArgoLOOM/tools/cosmo/API_SASHIMI_subhalos.py
	\end{lstlisting}
	
	This script:
	
	\begin{itemize}
		\item Reads parameters from env variable:
		\texttt{ARGOLOOM\_SASHIMI\_PARAMS\_JSON}
		\item Calls the adapter.
		\item Returns JSON output.
	\end{itemize}
	
	---
	
	\section{Edits to \texttt{argo-loom.py}}
	
	\subsection{Add Tool Schema}
	
	Add to \texttt{build\_tools\_schema()}:
	
	\begin{lstlisting}
		{
			"type": "function",
			"function": {
				"name": "sashimi_subhalos",
				"description": "Run SASHIMI-SI subhalo catalog generation.",
				"parameters": {
					"type": "object",
					"properties": {
						"M0": {"type":"number"},
						"redshift":{"type":"number"},
						"logmamin":{"type":"integer"},
						"sigma0_m":{"type":"number"},
						"w":{"type":"number"}
					},
					"required":["M0","sigma0_m","w"]
		}}}
	\end{lstlisting}
	
	---
	
	\subsection{Add Tool Execution Block}
	
	Inside \texttt{\# ---- EXECUTE TOOLS ----}:
	
	\begin{lstlisting}
		elif fname == "sashimi_subhalos":
		try:
		result = run_tool_sashimi_subhalos(fargs)
		except Exception as e:
		result = {"error": str(e)}
		
		messages.append({
			"role": "tool",
			"tool_call_id": tc.id,
			"name": "sashimi_subhalos",
			"content": json.dumps(result),
		})
	\end{lstlisting}
	
	---
	
	\section{Running the Agent}
	
	\begin{lstlisting}
		conda activate sashimi-si
		
		cd /Users/<username>/Documents/GitHub/ArgoLOOM
		
		python argo-loom.py --model gpt-4o
	\end{lstlisting}
	
	Example prompt:
	
	\begin{lstlisting}
		Run sashimi subhalos for host mass 1e12 Msun at z=0
		with sigma0_m=147.1 and w=24.33.
	\end{lstlisting}
	
	---
	
	\section{Expected Output}
	
	\begin{itemize}
		\item Tool call executed.
		\item SASHIMI progress printed.
		\item Agent returns summary:
	\end{itemize}
	
	\begin{lstlisting}
		N_filtered = 1353384
		Quantiles [1,10,50,90,99] = [...]
	\end{lstlisting}
	
	---
	
	\section{Cross-Checks}
	
	\begin{lstlisting}
		python - <<'PY'
		import json
		p=".../summary.json"
		print(json.load(open(p)))
		PY
	\end{lstlisting}
	
	---
	
	\section{Troubleshooting}
	
	Common issues encountered:
	
	\begin{itemize}
		\item \texttt{ModuleNotFoundError: numexpr}
		\item Missing \texttt{scipy.integrate.simps}
		\item Missing \texttt{openai}
		\item Missing \texttt{sentence\_transformers}
		\item FAISS installation issues
		\item API quota errors
	\end{itemize}
	
	\begin{tcolorbox}
		\textbf{Practical advice:}  
		Copy the error into ChatGPT and ask for resolution.  
		This was highly effective during development.
	\end{tcolorbox}
	
	---
	
	\section{Conclusion}
	
	We have documented a successful integration of:
	
	\begin{itemize}
		\item Cosmological SIDM simulation (\texttt{SASHIMI-SI})
		\item Agentic orchestration framework (\texttt{ArgoLOOM})
	\end{itemize}
	
	This enables automated generation, filtering, and reasoning over subhalo populations within the ArgoLOOM pipeline.
	
	---
	
\end{document}
